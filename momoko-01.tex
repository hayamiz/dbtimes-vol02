% -*- coding: utf-8 -*-

\chapter{データマイニングと機械学習批評}
% * (アスタリスク)付きの \chapter* コマンドは原則不可とする

\begin{flushright}
 早水 桃子 % ペンネーム
\end{flushright}

\begin{spacing}{0.6}
\noindent
{\footnotesize{本稿は、別にデータサイエンスや統計学の専門家ではない筆者が
The Database Timesのために
気儘に執筆し、無責任に寄稿したものである。
本稿で述べる内容は、著者名・書名が本文中に明示的に引用されていない限りは全て筆者の
個人的な意見であり、一般的な見解とは必ずしも一致しないということを予め宣言しておく。}}
\end{spacing}
 
\section{ブームの功罪}
『解体新書』は、江戸時代の医師たちの試行錯誤によって『ターヘル・アナトミア』というオランダ語の解剖学書から翻訳され、1774年に出版された歴史的書物である。まだオランダ語の辞書すらなかった時代であることを考えると、本文・図表合わせて五冊から成る医学書をわずか数年で翻訳・出版したというだけで大変な偉業に違いない。しかし、この本の功績は、当時鎖国政策をとっていた日本に初めて西洋の医学知識を伝導し、それまでの日本の医学の常識を覆したところにある。
『解体新書』以前の日本においては解剖はほとんど行われていなかったため、人体は「五臓六腑」という東洋医学の枠組みの中で捉えられていた。現代のように、人体の様々な臓器の構造と機能を体系的に捉えることは、この『解体新書』から始まったのである。

『解体新書』はのちに医学にとどまらずヨーロッパで発展した様々な学問から新しい知識や技術を吸収するという「蘭学(洋学)」ブームの契機になり、近代日本科学史に金字塔を打ち立てることとなった。
しかし、『解体新書』の翻訳者の一人である杉田玄白は、彼が85歳で亡くなる2年前に著した『蘭学事始』(1815年)という回想録の冒頭において、このブームを少々冷めた視線で眺めている。
%蘭学草創期の出来事を正しく伝えるために、85歳で亡くなる2年前の1815年に『蘭学事始』という回想録を著し、その冒頭でこのように述べている。

\begin{quote}
今時(きんじ)、世間に蘭学といふ事専ら行(おこな)はれ、志を立つる人は篤く学び、無識なる者は漫(みだ)りにこれを誇張す。
\end{quote}

ブームには、功罪があるものだ。昨今のデータサイエンスをとりまく流行も、恐らく例外ではない。『解体新書』が現代医学への道を切り拓き、やがて蘭学が社会現象の域にまで達したように、
初の網羅的な解剖学アトラスのような『パターン認識と機械学習』の大ヒットもまた、機械学習・データマイニングといった分野の大流行を物語る。
ジョン・ネイスビッツは『メガトレンド』(1982年)という本の中で「私たちは情報の海に溺れ、知識に飢えている」と見事に評したが、
%We are drowning in information but starved for knowledge. 
「情報爆発」や「ビッグデータ」へのソリューションが声高に語られ、それに煽られるかのようにバズワードに群がる烏合の衆は今日も騒がしいツイートをやめない。
無論、目新しい分野に期待を寄せるのは当然あるべき好奇心に違いないのだが、
パターン認識と機械学習の学習\footnote{特定の書籍に言及しているわけではない、と白々しくトボケておこう。}をしなければ何となく乗り遅れたような気分にさせられたり、とりあえず機械学習で何かをすればそれが何であっても人目を引くことが出来たり、データの海に溺れる者にデータマイニングを謳った藁を差し伸べる有象無象の輩が現れるのは、花形となった最先端の研究分野の熱気と言うよりはむしろ、ゴールドラッシュ\footnote{金が採掘された場所に一攫千金を目論む採掘者が殺到すること。1848年にカリフォルニアの川で砂金が発見されたことによるカリフォルニア・ゴールドラッシュが有名であるが、金の鉱脈が見つかるたびに世界各地で幾度となく繰り返し見られる現象である。}を彷彿とさせる熱病と呼ぶべきものである。

そのような現状を踏まえ、本稿では、「シラケつつノリ、ノリつつシラケる」\footnote{浅田彰の『構造と力―記号論を超えて』(1983年)のあまりにも有名な一節。}といった姿勢で、幾つかの重要な点について批判的に考察したいと思う。

第一の懸念は、データサイエンスをとりまく熱気が単なる空騒ぎに浪費され、現実の問題解決に対する理論や技法の有用性や意義を見ることないまま

大きな期待が集団的な失望へ変わると同時にブームの終焉を(最悪の場合は学問分野の衰退)を招きうるのではないかということである。

第二の懸念は、社会を助ける理論や技術があっても、それが社会に受け入れられるかどうかは別問題ということである。
\begin{itemize}
\item データマイニングとはいったい何なのか
\item 機械学習の応用研究で見えてくる機械学習の限界とは何か
\item 結局機械学習は人間にとって重宝されるものなのか、重宝されるならばどのように活かされうるのか
\end{itemize}