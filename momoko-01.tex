% -*- coding: utf-8 -*-

\chapter{メルトダウンする機械学習(仮)}
% * (アスタリスク)付きの \chapter* コマンドは原則不可とする

\begin{flushright}
 早水 桃子 % ペンネーム
\end{flushright}

\begin{spacing}{0.6}
\noindent
{\footnotesize{本稿は、別にデータサイエンスや統計学の専門家ではない筆者が
The Database Timesのために
気儘に執筆し、無責任に寄稿したものである。
本稿で述べる内容は、著者名・書名が本文中に明示的に引用されていない限りは全て
筆者の個人的な意見であり、一般的な見解とは必ずしも一致しないということを予め宣言しておく。}}
\end{spacing}
 
\section{ブームの功罪}
『解体新書』は、江戸時代の医師たちの試行錯誤によって『ターヘル・アナトミア』というオランダ語の解剖学書から翻訳され、1774年に出版された歴史的書物である。まだオランダ語の辞書すらなかった時代であることを考えると、本文・図表合わせて五冊から成る医学書をわずか数年で翻訳・出版したというだけで大変な偉業である。しかし、この本の功績は、当時鎖国政策をとっていた日本に初めて西洋の医学知識を伝導し、それまでの日本の医学の常識を覆したところにある。
『解体新書』以前の日本においては解剖はほとんど行われていなかったため、人体は「五臓六腑」という東洋医学の概念によって捉えられていた。今でこそ我々は人体を色々な臓器の解剖学的な構造と生理学的な機能が作るシステムとして捉えているが、この枠組みを日本に初めてもたらしたのは、『解体新書』に他ならない。

『解体新書』は医学にとどまらず、ヨーロッパで発展した様々な学問から新しい知識や技術を吸収するという「蘭学(洋学)」ブームの契機になり、近代日本科学史に金字塔を打ち立てることとなった。
しかし、『解体新書』の翻訳者の一人である杉田玄白は、彼が85歳で亡くなる2年前に著した『蘭学事始』(1815年)という回想録の冒頭において、このブームを少々冷めた視線で眺めている。
%蘭学草創期の出来事を正しく伝えるために、85歳で亡くなる2年前の1815年に『蘭学事始』という回想録を著し、その冒頭でこのように述べている。

\begin{quote}
今時(きんじ)、世間に蘭学といふ事専ら行(おこな)はれ、志を立つる人は篤く学び、無識なる者は漫(みだ)りにこれを誇張す。
\end{quote}

ブームには、功罪があるものだ。昨今のデータサイエンスをとりまく流行も、恐らく例外ではない。『解体新書』が現代医学への道を切り拓き、やがて蘭学が社会現象の域にまで達したように、
初の網羅的な解剖学アトラスのような『パターン認識と機械学習』の大ヒットもまた、機械学習・データマイニングといった分野の大流行を物語る。
ジョン・ネイスビッツは『メガトレンド』(1982年)という本の中で「私たちは情報の海に溺れ、知識に飢えている」と見事に評したが、
%We are drowning in information but starved for knowledge. 
「情報爆発」や「ビッグデータ」へのソリューションが声高に語られ、それに煽られるかのようにバズワードに群がる烏合の衆は今日も騒がしいツイートをやめない。

無論、目新しい分野に期待を寄せるのは当然あるべき好奇心に違いないのだが、
パターン認識と機械学習の学習\footnote{特定の書籍に言及しているわけではない、と白々しくトボケておく。}をしなければ何となく乗り遅れたような気分にさせられたり、とりあえず機械学習で何かをすればそれが何であっても人目を引くことが出来たり、データの海に溺れる者にデータマイニングを謳った藁を差し伸べる有象無象の輩が現れるのは、花形となった最先端の研究分野の熱気と言うよりはむしろ、ゴールドラッシュ\footnote{金が採掘された場所に一攫千金を目論む採掘者が殺到すること。1848年にカリフォルニアの川で砂金が発見されたことによるカリフォルニア・ゴールドラッシュが有名であるが、金の鉱脈が見つかるたびに世界各地で幾度となく繰り返し見られる現象である。}を彷彿とさせる熱病と呼ぶべきものである。

賢明な読者諸兄姉ならば、このお祭り騒ぎに付和雷同してはいられない程度の危機感があるはずだ。データサイエンスをとりまく熱気は単なる空騒ぎに浪費されるかもしれないし、これまでに提案されてきた数々の理論や技法は、現実の問題解決に対する有用性や意義が不明瞭なまま忘れ去られ、それを実装したコードも無用の長物となるかもしれない。データ解析に求められるソフトウェア・ハードウェアに投資しても望む結果が何も得られなかった場合、ユーザーの少々無理のある過剰な期待はやがて失望に変わり、それが集団的な失望へ変わる時にブームは終焉を迎え、最悪の場合は学問分野そのものが廃れていくかもしれない。
また、新しい考えや技術は受け入れられるどころか、脅威と見なされて排除されることも珍しくない。実際、蘭学ブームは蘭方医学と漢方医学の深刻な対立を招き、蘭学は蘭書翻訳取締令などによって思想弾圧の憂き目を見ることになった。別の例で言えば、産業革命は機械による自動化・作業効率化をもたらした一方で、公害という新たな社会問題を生むことになり、さらに失業を恐れた労働者たちによる機械破壊運動が深刻化することになった\footnote{この機械打ち壊し運動(ラッダイト運動)になぞらえて、雇用を不安定にするIT化・自動化に反対する思想はネオ・ラッダイトと呼ばれている。}。人間を助けうる技術が世に出たところで、人間がそれを欲しない限り、その技術の真価は発揮されず、認められないままとなる。

だからといって、その危機感は時代の潮流に乗ることを放棄するものではない。本稿における我々の基本的なスタンスは「シラケつつノリ、ノリつつシラケる、これである」\footnote{浅田彰の『構造と力―記号論を超えて』(1983年)のあまりにも有名な一節。}。このスタンスに反するのは、持て囃されている理論や技術を盲目的に有り難がったり、その中身を理解するだけで満足してしまうことである。どんなに高尚なものであっても、現実世界の人間にとっては道具にすぎないし、道具の長所や短所、その活用のしどころ、その道具のより良いあり方というのは道具を見ているだけでは分からないものだ。道具を実際に使ってみることで初めてこれらを語ることができるのと同様に、アルゴリズムの良し悪し、機械学習が解決をもたらす問題、機械学習が人間と共生するためにどうあるべきかというのは機械学習の応用研究なしには絶対に語れないものである。

\section{データマイニングとはいったい何なのか}
データマイニング
%\section{機械学習の応用研究で見えてくる機械学習の限界とは何か}
%\section{結局機械学習は人間にとって重宝されるものなのか、重宝されるならばどのように活かされうるのか}