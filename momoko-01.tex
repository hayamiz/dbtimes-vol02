% -*- coding: utf-8 -*-

\chapter{機械学習の応用とオープンデータベースの価値}
% * (アスタリスク)付きの \chapter* コマンドは原則不可とする

\begin{flushright}
 早水桃子 % ペンネーム
\end{flushright}

\section{機械学習ブームの功罪}

『解体新書』は、江戸時代の医師たちの試行錯誤によって『ターヘル・アナトミア』というオランダ語の解剖学書から翻訳され、1774年に出版された歴史的書物である。まだオランダ語の辞書すらなかった時代であることを考えると、本文・図表合わせて五冊から成る専門書をわずか数年で翻訳・出版したというだけで大変な偉業に違いない。しかし、この本が近代日本科学史の金字塔と位置づけられているのは、当時鎖国政策をとっていた日本に初めて西洋の医学的知識を伝導し、医学にとどまらずヨーロッパからの新しい知識や技術を翻訳して吸収し、概念化・体系化するという「蘭学(洋学)」ブームの契機になったというところが大きいだろう。

『解体新書』の翻訳者の一人である杉田玄白は、蘭学草創期の出来事を正しく伝えるために、85歳で亡くなる2年前の1815年に『蘭学事始』という回想録を著し、その冒頭でこのように述べている。

\begin{quote}
今時(きんじ)、世間に蘭学といふ事専ら行(おこな)はれ、志を立つる人は篤く学び、無識なる者は漫(みだ)りにこれを誇張す。
\end{quote}

専門家が集い、適切な訳語を探りながら翻訳された『パターン認識と機械学習』という本はこの分野のブームの幕開けになったと言っても過言ではない。今や「機械学習」という言葉は「データマイニング」や「ビッグデータ」などとともにバズワードと化しており、様々な分野の人々の関心と期待を集めている。その一方で、機械学習は何が出来るのか、また何をすべきなのか、といった意義や有用性は不明瞭になりやすく、「よく分からないけれど機械学習は勉強しなくてはいけないものだ」とか「何が出来るのか分からないけれど、機械学習を使って何かがしたい」などと考える向きも少なくないのではないだろうか。

\section{実りあるブームのために}
