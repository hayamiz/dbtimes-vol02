% -*- coding: utf-8 -*-


\chapter{IPv4がこの先生きのこるには}
% * (アスタリスク)付きの \chapter* コマンドは原則不可とする

\begin{flushright}
 yuyarin % ペンネーム
\end{flushright}

\section{はじめに}

\lettrine{イ}
ンターネットの主要技術であるIPv4というプロトコル。その中で個体識別子として使われるIPv4アドレスは32bitで表現され数は約42億個である。
かつては「無限アドレスだぜヒャッハー」と湯水のように消費されていたのだが,70億もの人間が社会インフラとしてインターネットを必要としている
今となっては,このアドレスの数が全く足りなくなってしまっているのだ。
レジストリから事業者へのIPv4アドレスの新規割り振りはほぼ終了しており,IPv4アドレスは「枯渇」したと表現されている。
