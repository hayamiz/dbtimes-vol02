% -*- coding: utf-8 -*-

%% ======
%%  Font
%% ======

\renewcommand{\sfdefault}{phv}

%% =============
%%  Page Layout
%% =============

% B5: 182mm x 257mm
\setlength{\voffset}{0mm}
\setlength{\topmargin}{-23mm}
\setlength{\headsep}{25pt}
\setlength{\textheight}{37\Cvs}
\setlength{\textwidth}{\fullwidth}
\setlength{\footskip}{10mm}

% set margin for openleft
\setlength{\evensidemargin}{\oddsidemargin}
%\addtolength{\evensidemargin}{-\textwidth}

% \setlength{\oddsidemargin}{-\oddsidemargin}
% \setlength{\evensidemargin}{-\oddsidemargin}


%% ===================
%%  Header and Footer
%% ===================

\pagestyle{fancy}

\fancyhead[LO]{}
\fancyhead[RO]{
\begin{picture}(0,0)(0,0)
 \linethickness{12pt}
 \put(15,3.5){\line(1,0){80}}
 \linethickness{0pt}
 \put(15,3.5){\framebox(20,0){\textcolor{white}{\fontfamily{ugq}\selectfont\small \thepage}}}
\end{picture}
}
\fancyhead[LE]{
\begin{picture}(0,0)(0,0)
 \linethickness{12pt}
 \put(-92,3.5){\line(1,0){80}}
 \linethickness{0pt}
 \put(-32,3.5){\framebox(20,0){\textcolor{white}{\fontfamily{ugq}\selectfont\small \thepage}}}
\end{picture}
}
\fancyhead[RE]{}
\fancyfoot{}
\renewcommand{\headrulewidth}{0.0pt}
\renewcommand{\headrule}{}

\fancypagestyle{plainhead}{
\fancyhead[LO]{}
\fancyhead[RO]{
\begin{picture}(0,0)(0,0)
 \linethickness{12pt}
 \put(15,3.5){\line(1,0){80}}
 \linethickness{0pt}
 \put(15,3.5){\framebox(20,0){\textcolor{white}{\fontfamily{ugq}\selectfont\small \thepage}}}
\end{picture}
}
\fancyhead[LE]{
\begin{picture}(0,0)(0,0)
 \linethickness{12pt}
 \put(-92,3.5){\line(1,0){80}}
 \linethickness{0pt}
 \put(-32,3.5){\framebox(20,0){\textcolor{white}{\fontfamily{ugq}\selectfont\small \thepage}}}
\end{picture}
}
\fancyhead[RE]{}
\fancyfoot{}
\renewcommand{\headrulewidth}{0.0pt}
\renewcommand{\headrule}{}
}

\fancypagestyle{fronthead}{
\fancyhead[LO]{}
\fancyhead[RO]{}
\fancyhead[LE]{}
\fancyhead[RE]{}
\cfoot{-- \thepage --}
\renewcommand{\headrulewidth}{0.0pt}
\renewcommand{\headrule}{}
}


%% ====================
%%  Customized chapter
%% ====================


\makeatletter
\renewcommand{\chapter}{%
  \if@openright\cleardoublepage\else\clearpage\fi
  \plainifnotempty % 元: \thispagestyle{plain}
  \global\@topnum\z@
  \if@english \@afterindentfalse \else \@afterindenttrue \fi
  \secdef\@chapter\@schapter}
\def\@chapter[#1]#2{%
  \refstepcounter{chapter}%
  \addcontentsline{toc}{chapter}{#1}
  \chaptermark{#1}%
  \addtocontents{lof}{\protect\addvspace{10\p@}}%
  \addtocontents{lot}{\protect\addvspace{10\p@}}%
  \if@twocolumn
    \@topnewpage[\@makeschapterhead{#2}]%
  \else
    \@makeschapterhead{#2}%
    \@afterheading
  \fi}
\def\@schapter#1{%
  \chaptermark{#1}%
  \if@twocolumn
    \@topnewpage[\@makeschapterhead{#1}]%
  \else
    \@makeschapterhead{#1}\@afterheading
  \fi}

\def\@makeschapterhead#1{%
  {\parindent \z@ \raggedright
    \normalfont
    \interlinepenalty\@M
	\begin{flushright}
	 \begin{minipage}{0.9\textwidth}
	 {\LARGE \headfont #1}
	 \end{minipage}
	\end{flushright}
	\vskip-3\Cvs\includegraphics[width=10.0mm]{images/arrow.eps}\par\nobreak\noindent
    \vskip 1.8\Cvs}} % 欧文は40pt

% section
% 後アキの調整
\renewcommand{\section}{%
  \if@slide\clearpage\fi
  \@startsection{section}{1}{\z@}%
  {\Cvs \@plus.5\Cdp \@minus.2\Cdp}% 前アキ
  {.7\Cvs \@plus.3\Cdp}% 後アキ
  {\normalfont\Large\headfont\raggedright}}

% 前・後アキの調整
\renewcommand{\subsection}{\@startsection{subsection}{2}{\z@}%
  {0.7\Cvs \@plus.5\Cdp \@minus.2\Cdp}% 前アキ
  {.25\Cvs \@plus.3\Cdp}% 後アキ
  {\normalfont\large\headfont}}
\makeatother

\renewcommand{\prechaptername}{}
\renewcommand{\postchaptername}{}

\makeatletter
\renewcommand{\thesection}{\S\,\@arabic\c@section}
\makeatother

\ifx\Cht\undefined
 \newdimen\Cht\newdimen\Cdp
 \setbox0\hbox{\char\jis2121}\Cht=\ht0\Cdp=\dp0\fi
\makeatletter
\long\def\linespace#1#2{\par\noindent
  \dimen@=\baselineskip\multiply\dimen@ #1\advance\dimen@-\baselineskip
  \advance\dimen@-\Cht\advance\dimen@\Cdp
  \setbox0\vbox{\noindent #2}\advance\dimen@\ht0\advance\dimen@-\dp0%
  \vtop to\z@{\hbox{\vrule width\z@ height\Cht depth\z@
   \raise-.5\dimen@\hbox{\box0}}\vss}%
  \dimen@=\baselineskip\multiply\dimen@ #1\advance\dimen@-\baselineskip
  \vskip\dimen@}
\makeatother



\makeatletter %引用スタイルの変更
\def\@cite#1{\textsuperscript{#1)}}
\makeatother
\makeatletter %参考文献リストの括弧を変更
\def\@cite#1{\textsuperscript{#1)}}
\renewcommand*{\@biblabel}[1]{#1)\hfill}%
\makeatother


%% Listings
\newcommand{\listingsize}{\small}
\lstset{language=SQL,
morekeywords={RETRIEVE,RANGE,OF,IS},
numbers = left,
numberstyle = {\tiny},
numbersep = 5pt,
breaklines = false,
breakindent = 40pt,
flexiblecolumns = true,
keepspaces = false,
basicstyle = \normalsize,
identifierstyle = \itshape\listingsize,
commentstyle = \fontfamily{ptm}\selectfont\listingsize,
stringstyle = \upshape\listingsize,
tabsize = 4,
escapechar = |,
}
