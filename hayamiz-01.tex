% -*- coding: utf-8 -*-

\chapter{Hadoopつかったら負けかなと思ってる}

\begin{flushright}
はやみず
\end{flushright}

2009年,Web上にある記事が公開されて大きな話題を呼んだ。
``MapReduce: A major step backwards'' と題されたその記事は,
当時のクラウドコンピューティングブームを原動力として大流行していたMapReduceを痛烈に批判するものだった。
日本語にするならば「MapReduce: 大きな技術的後退」となるだろうか。
タイトルからして挑戦的であるが,その中身もかなり手厳しい。

記事の中に要点のリストがあるので,拙訳とともに引用する:

\begin{quote}
\begin{enumerate}
 \item A giant step backward in the programming paradigm for large-scale data intensive applications \\
	   大規模データインテンシブアプリケーションにおけるプログラミングパラダイムの大いなる後退
 \item A sub-optimal implementation, in that it uses brute force instead of indexing \\
	   索引を使わない総当り方式の,およそ最善とは言いがたい実装
 \item Not novel at all -- it represents a specific implementation of well known techniques developed nearly 25 years ago \\
	   全く新規性に欠ける -- 25年近く前に開発された手法の限定的な実装でしかない
 \item Missing most of the features that are routinely included in current DBMS \\
	   現代のDBMSがごく当たり前に持っている機能の殆どが欠けている
 \item Incompatible with all of the tools DBMS users have come to depend on \\
	   DBMSユーザが使っているツールのほとんど全てと互換性がない
\end{enumerate}
\end{quote}

この記事を書いた人は誰かというと,
データベース界の超大御所 David J. DeWitt と Michael Stonebraker \footnote{RDBMSを世界で最初に実装したStonebrakerの功績は The Database Times vol.1 を参照のこと}\footnote{ちなみにこの2人とも,DB界で最も権威ある SIGMOD Edgar F. Codd 賞の受賞者である} である。
そんな2人がこれだけけちょんけちょんに言うのだから,MapReduceはよっぽど酷いもののように思えるかもしれない。
事実,この記事の指摘はある側面ではかなり的を得ている。

しかし一方で,この記事は DeWitt と Stonebraker の壮大な負け惜しみでもあるのではないか,と私は思っている。

\section{データベースの歴史はプログラマ解放戦争だった}

Stonebrakerはリレーショナルデータベース(RDB)研究の先駆者であるが,
その大元となるアイディアであるリレーショナルデータモデルは Edgar F. Codd により提唱されたものである。

リレーショナルデータモデルの一番の目的は,
データベースアプリケーションに関わるプログラマを``手続き''から解放することであった。

Codd がリレーショナルデータモデルを提唱した時代には,
データベースといえば階層型データベースと呼ばれるものが主流であった。
階層型データベースでは,データが木構造で階層的に格納されており,
データベースアプリケーションを作成する際には,
プログラマが階層をたどる処理を1つ1つ実装してゆかなければならなかった。
データの種類が追加されたときには,データを格納する階層構造も変化する。
そのため,データベースアプリケーションのコードを新たな階層構造に合わせて修正してやる必要があった。


\section{Big Dataの波に乗って}

