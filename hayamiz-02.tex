% -*- coding: utf-8 -*-

\chapter{PgDay 2012 Japan レポート}

\begin{flushright}
 {\headfont 著:はやみず}
\end{flushright}

2012年11月30日にNPO法人日本PostgreSQLユーザ会(JPUG)主催で開催されたPgDay 2012 Japanに参加してきました。

\section{開催概要}

日本におけるPgDayの名を冠するカンファレンスの開催は,今回が初めてだったようです。
JPUGの主催するPostgreSQLの大規模なカンファレンスは,
これまでは毎年2〜3月頃にPostgreSQL Conferenceが開催されていたのですが,
年度末で都合がつかずに参加できないという声が多いために,
秋開催へと移行することになっています。
そのため,今年は秋開催移行への予行演習的な意味も含めて(?)
2012年2回目のカンファレンスであるPgDay 2012 Japanが開催されました。

セミナーチケット140名分は早々と売り切れ,
追加分も販売され,当日の参加者数は222名(スタッフ発表)ということでした。
10月末に開催されたPG Conference Europe 2012では,
参加者がヨーロッパ全体から集まる4日間のカンファレンスで計290名参加であったことを考えると,
日本におけるPostgreSQL人気の高さ,
そしてユーザグループの成熟度がいかに高いかを伺うことができます。

\begin{figure}[tb]
 \begin{center}
  \includegraphics[width=12cm]{images/pgday-water.eps}\par
  \noindent 
{\bf ▲受付を済ませると,
  発表資料に加えてJPUGイベントではお馴染みの「かめすい」がもらえる}
 \end{center}
\end{figure}

\section{``You are PostgreSQL!''}

PgDay当日は,PostgreSQL開発チームのコアチームであるMagnus Hagander氏の基調講演で幕を開けました。
Hagander氏は ``Inside the PostgreSQL community - How the product gets built'' という題で,
PostgreSQLコミュニティにおける様々な人々の立ち位置や,
彼らがどのように協力しあってPostgreSQLを作り上げているのか,
そしてどのようにしてPostgreSQLの開発に参加できるか,
ということを紹介してゆきました。

この講演で一番強いメッセージとして感じたのは,
PostgreSQLは特定の企業には属しておらず,
そしてその開発は特定の企業や個人によってコントロールされるものではなく,
各々がコミュニティに参加して協力することで開発が進められている,
ということでした。
もちろんPostgreSQLでもEnterpriseDBや2ndQuadrant,
また日本からはNTTデータやSRA OSSなどの企業がPostgreSQLの開発に関わってはいるものの,
あくまでコミュニティに参加することを通して協力しながら開発を進めていることは,
他のDBMSとは一線を画します。
このような,様々な企業や個人が協力しながらコミュニティとして開発を進めていくことができる仕組み,
そして現在のPostgreSQLコミュニティを形作る人たちこそが,
継続的なOSSの進化・発展に不可欠な存在なのです。
つまり,ある意味PostgreSQLというプロダクトそのものよりも,
PostgreSQLというコミュニティのほうが重要であるというのが,
この講演における最大のメッセージと言えます。

そしてHagander氏のもうひとつの大きなメッセージは,
「今日この場にいる皆さん自身もPostgreSQLコミュニティの一員である(``You are PostgreSQL.'')」
ということでした。
PgDayに集まってきた参加者は,皆PostgreSQLに興味をもち,実際に使っている人たちであり,
PostgreSQLコミュニティの一員なのだ,ということです。
そして,更にPostgreSQLコミュニティの一員として積極的な参加を促すために,
Hagander氏は様々な形でPostgreSQLの開発に関わる方法を紹介してゆきました。
最も直接的にはパッチを書くという方法で参加することもできますが,
それ以外にもドキュメントを書いたり,翻訳したりという作業も「プロダクトを作る」
ということにおいては非常に重要であり,そして多くの人手を必要としています。
また,ユーザグループを作ったり,
PostgreSQLのことをブログに書くだけでも,
それが即ちPostgreSQLのマーケティング活動になるため,
価値あるPostgreSQLコミュニティ参加方法となります。
特に,最近では英語圏以外でもPostgreSQLに関する様々な情報発信が行われるようになっているため,
記事や書籍を英語に``逆輸入''することで,
世界中の人と知見を共有することができます。
このように,コードが書ける人も,書けない人も,
実に様々な側面からPostgreSQLコミュニティに参加することができる,
だから簡単なことから少しずつコミュニティに参加してみよう,
というHagander氏の熱い想いがこもった講演でした。

\section{注目は高可用データベースシステム}

午後からのプログラムは,PostgreSQLクラスタ特集といっても過言ではないほど,
レプリケーションや高可用技術に関する発表が中心でした。
なかでも,PostgreSQLのレプリケーション機能を中心とした高可用システムの解説や,
そのソリューションパッケージの紹介などが目立ちました。

PostgreSQLはこの数年,レプリケーション機能の充実に非常に力を入れています。
PostgreSQLでは,バージョン9.0から初めてレプリケーション機能が利用できるようになりました。
2010年にリリースされたPostgreSQL 9.0はもともと8.5となるはずでした。
しかし,レプリケーションとホットスタンバイという重大な2つの機能が新たに加わるインパクトを表すために,
あえてメジャーバージョンを増やして9.0になったという経緯があります。
このことからも,いかにレプリケーションが待ち望まれていたかがわかります。
その後も2011年にリリースされた PostgreSQL 9.1 では同期レプリケーション機能が,
2012年にリリースされたPostgreSQL 9.2ではカスケードレプリケーション機能が導入され,
用途・目的に応じて様々なレプリケーション方法を利用することができるようになりました。

PostgreSQLでレプリケーション機能が求められる大きな理由が,
業務用途,とくにサービスの停止時間ができるだけ短い,
高可用性データベースシステムにおいてPostgreSQLを使えるようにしたい,
という要望です。
これに応えるべくレプリケーション機能が進化したPostgreSQLの業務利用は確実に広まりをみせており,
日本国内でもその傾向は明らかに見て取れます。
JPUGの主催するPostgreSQLカンファレンスでは参加者数が年々増加しており,
また今秋にはPostgreSQLの業務システム適用を促進するためのPostgreSQLエンタープライズコンソーシアム(PGECons)が設立されました。

そのような背景もあり,
今年3月に開催されたPostgreSQL Conference 2012 Japanでも
レプリケーションや高可用システムの発表が目立ったのですが,
今回は更にその流れが色濃く反映されたプログラムとなっていました。
午後のプログラムはPostgreSQLのレプリケーション機能を紹介する講演,
そしてレプリケーション機能をベースとした高可用クラスタの構成に関するノウハウの講演が行われた後,
メイントラックとチュートリアルトラックに別れて,
メイントラックでは3本続けてPostgreSQLクラスタに関する講演が行われました。

1つめの講演では,NTT OSSセンタの

\section{PostgreSQLへの理解を深めるチュートリアルトラック}

午後の2トラック構成では,
メイントラックでは注目のPostgreSQLクラスタに関する話題を取り扱っていた一方で,
チュートリアルトラックではPostgreSQL初級〜中級者がよりPostgreSQLを理解するために,
PostgreSQL内部構成,システム設計の勘所,
そしてクエリプランの読み方のそれぞれについて深く掘り下げた講演が行われました。

その中でも特に個人的に面白かったのは,
NTT OSSセンタの坂本 昌彦氏による「実践!PostgreSQL 運用」です。
この講演では,運用そのものではなく,
運用時に問題が発生しないようなPostgreSQLシステムをいかに設計するかについて,
設計時のチェックポイントを紹介する形で行われました。
坂本氏曰く「とりあえずPostgreSQLを使えるのは当然として,
``プロ''としてプライドがある人ならばこのくらいできて欲しいよね,
という話をします」
という言葉通り,NTTにおける豊富なPostgreSQL運用経験をもとにして,
非常に具体的なところまで踏み込んだシステム設計ノウハウが説明されており,
こんな詳細なところまで喋っちゃっていいの!?
というくらい,PostgreSQLを業務システムで使う上では役に立つ内容でした。
坂本氏はPostgreSQLの設計は大きく分けて TODO:

\section{気になる今後は}

今回のPgDayでは,PostgreSQL界隈で今まさにホットなレプリケーションが話題の中心で,
今後のPostgreSQLについてはほとんど語られていませんでした。
その部分について,現状で明らかになっているものに少しばかり触れてみましょう。

Hagander氏の基調講演の中で,「新たな機能を提案するときは,
『ほかのDBMSにもあるから○○をPostgreSQLでも欲しい』
というのはやめてくれ」と言ってはいたものの,
今秋にリリースされたPostgreSQL 9.2の新機能や,
9.3でリリースが見込まれている新機能の情報を眺めていると,
やはり他のDBMSと比べて欠けているものを補おうとしている側面は見受けられます。
特に,EnterpriseDB社\footnote{EnterpriseDB社にはコアチームメンバを始めとしたPostgreSQL開発の中心的人物が数多く所属している。}ではPostgreSQLの機能強化版であるPostgreSQL Plusを販売していますが,
PostgreSQL Plusの大きな特徴はOracleとの互換性であることからも,
Oracleと比べて足りない機能はPostgreSQL開発ロードマップの重要な指針であることは間違いありません。
PostgreSQL 9.2の新機能であるIndex Only Scanは,
OracleやMySQLでは既に利用可能な機能であり,以前より強く待ち望まれていました。
\footnote{ちなみに,IBMのDB2では索引を張っていないカラムの集計もIndex Only Scanで出来るよう,
索引に指定した属性の値を同居させる機能を持っていたりします}
また,PostgreSQL 9.0から次々に強化されるレプリケーション機能や,
Postgres-XCなどは明らかにOracle RACを意識したものでしょう。
PostgreSQL 9.3では,Materialized Viewが実装される見通しです。
これも商用DBでは既に一般的に提供されており,
PostgreSQLユーザからの要望調査第1位にある機能です\footnote{\url{http://postgresql.uservoice.com/forums/21853-general}}。
9.2でのレプリケーションはあくまでもマスタは1インスタンスに限られていますが,
9.3ではマルチマスタ機能の導入も検討されているようです。

一方でPostgreSQL独特の面白い機能として,
書込み可能なfdw (Foreign Data Wrapper)のようなものもあります。
fdwを使うと,CSVファイルなどの外部データを,あたかもPostgreSQLのテーブルかのように扱い,
SQLを発行することができます。
現状ではデータの読込みしかできないのですが,
これを書込みもできるようにする機能の実装が進んでおり,
バージョン9.3での導入に向けて開発MLでも活発に議論されています。

またPostgreSQLの開発そのものには関係ありませんが,
昨年から今年にかけてクラウドにおけるPostgreSQLの存在感が増してきています。
HerokuではDBaaSとして Heroku Postgres の提供を2011年末から開始しています。
最近では,データベースをgithubのレポジトリのようにforkできるという,
面白い機能もリリースされて話題を呼びました\footnote{\url{https://postgres.heroku.com/blog/past/2012/11/8/fork_your_data/}}。
さらに,2010年末にHerokuを買収したSalesforceが,
PostgreSQLのメーリングリストにPostgreSQL技術者を募集するメールを投稿したことから,
SalesforceがOracleからOSSへシフトするのか?
と話題を呼びました\footnote{\url{http://archives.postgresql.org/pgsql-jobs/2012-10/msg00003.php}}。
