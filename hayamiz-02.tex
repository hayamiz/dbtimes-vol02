% -*- coding: utf-8 -*-

\chapter{PgDay 2012 Japan レポート}

\begin{flushright}
 {\headfont 著:はやみず}
\end{flushright}

2012年11月30日にNPO法人日本PostgreSQLユーザ会(JPUG)主催で開催されたPgDay 2012 Japanに参加してきました。

\section{開催概要}

日本において,PgDayの名を冠するカンファレンスの開催は今回が初めてだったようです。
JPUGの主催するPostgreSQLの大規模なカンファレンスとしては,
これまでは毎年2〜3月頃にPostgreSQL Conferenceが開催されていました。
しかし,年度末で都合がつかずに参加できないという声が多いため,
秋開催へと移行することになっています。
今年は秋開催移行への予行演習的な意味も含めて(?)
11月にPgDay 2012 Japanが開催されました。

セミナーチケット140名分は早々と売り切れたため,追加分も販売されました。
その追加分も売り切れ,当日の参加者数は222名(スタッフ発表)ということでした。
ヨーロッパ全体のカンファレンスであるPG Conference Europe 2012の参加者数が290名であったことを考えると,
日本におけるPostgreSQL人気の高さだけでなく,
ユーザグループの成熟度の高さを伺うことができます。

% \begin{figure}[tb]
%  \begin{center}
%   \includegraphics[width=12cm]{images/pgday-water.eps}\par
%   \noindent 
% {\bf ▲受付を済ませると,
%   発表資料に加えてJPUGイベントではお馴染みの「かめすい」がもらえる}
%  \end{center}
% \end{figure}

\section{``You are PostgreSQL!''}

PgDay当日は,PostgreSQL開発コアチームメンバであるMagnus Hagander氏の基調講演で幕を開けました。
Hagander氏は ``Inside the PostgreSQL community - How the product gets built'' という題で,
PostgreSQLコミュニティにおける様々な人たちの立ち位置や,
彼らがどのように協力しあってPostgreSQLを作り上げているのか,
そしてこれからPostgreSQLの開発に参加するためにはどうすればよいのか,
ということを紹介してゆきました。

この講演の一番強いメッセージは,
PostgreSQLは特定の企業には属しておらず,
そしてその開発は特定の企業や個人によってコントロールされるものではなく,
各々がコミュニティに参加して協力することで開発が進められている,
ということです。
EnterpriseDBや2ndQuadrant,
また日本からはNTTデータやSRA OSSなどの企業がPostgreSQLの開発に関わってはいるものの,
あくまでコミュニティの一員として開発を進めていることに関して,
PostgreSQLは他のDBMSと一線を画しています。
様々な企業や個人が協力しながらコミュニティとして開発を進めていくことができる仕組み,
そしてコミュニティに参加する人たちこそが,継続的なOSSの進化・発展に不可欠な存在なのです。
その意味では,PostgreSQLというプロダクトそのものよりも,
PostgreSQLというコミュニティのほうが重要であるというのが,
この講演における最大のメッセージと言えます。

そしてHagander氏のもうひとつの大きなメッセージは,
「今日この場にいる皆さん自身もPostgreSQLコミュニティの一員である(``You are PostgreSQL.'')」
ということでした。
PgDayに集まってきた参加者は,皆PostgreSQLに興味をもち,実際に使っている人たちであり,
PostgreSQLコミュニティの一員なのだ,ということです。
またHagander氏は様々な形でPostgreSQLの開発に関わる方法を紹介してゆきました。
最も直接的にはパッチを書くという方法で参加することもできますが,
それ以外にもドキュメントを書いたり,翻訳したりという作業も「プロダクトを作る」
ということにおいては非常に重要であり,そして多くの人手を必要としています。
また,ユーザグループを作ったり,
PostgreSQLのことをブログに書くだけでも,
それが即ちPostgreSQLのマーケティング活動になるため,
価値あるPostgreSQLコミュニティ参加方法となります。
最近では英語圏以外でもPostgreSQLに関する様々な情報発信が行われるようになっているため,
記事や書籍を英語に``逆輸入''することで,
世界中の人と知見を共有することができます。
このように,コードが書ける人も書けない人も,
実に様々な側面からPostgreSQLコミュニティに参加することができる,
だから簡単なことから少しずつコミュニティに参加してみよう,
というHagander氏の熱い想いがこもった講演でした。

\section{注目は高可用データベースシステム}

午後からのプログラムは,PostgreSQLクラスタ特集といっても過言ではないほど,
レプリケーションや高可用技術に関する発表が中心でした。

PostgreSQLはこの数年,レプリケーション機能の充実に非常に力を入れています。
レプリケーション機能が初めて搭載されたのは,
2010年にリリースされたバージョン9.0です。
PostgreSQL 9.0はもともと8.5となるはずでした。
しかし,レプリケーションとホットスタンバイという重大な2つの機能が新たに加わるインパクトを表すために,
あえてメジャーバージョンを上げて9.0になったという経緯があります。
このことからも,いかにレプリケーションが待ち望まれていたかがわかります。
その後も2011年にリリースされた PostgreSQL 9.1 では同期レプリケーション機能が,
2012年にリリースされたPostgreSQL 9.2ではカスケードレプリケーション機能が導入され,
用途・目的に応じて様々なレプリケーション形態を利用することができるようになりました。

PostgreSQLでレプリケーション機能が求められる大きな理由が,
高可用性なデータベースシステム構築のためにPostgreSQLを使えるようにしたいという要望です。
PostgreSQLがレプリケーション機能を搭載してからは,
その業務利用は確実に広まりをみせており,
日本国内でも大きな盛り上がりを見せています。
そのことを示すように,
JPUGの主催するPostgreSQLカンファレンスでは参加者数が年々増加しています。
また今秋には,
PostgreSQLの業務システム適用を促進するためのPostgreSQLエンタープライズコンソーシアム(PGECons)が設立されました。

そのような背景もあり,
今年3月に開催されたPostgreSQL Conference 2012 Japanでも
レプリケーションや高可用システムの発表が目立ったのですが,
今回は更にその流れが色濃く反映されたプログラムとなっていました。
PgDayの午後のプログラムは,PostgreSQLのレプリケーション機能を紹介する講演,
そしてレプリケーション機能をベースとした高可用クラスタの構成に関するノウハウの講演が行われた後,
メイントラックとチュートリアルトラックに別れて,
メイントラックではなんと3本続けてPostgreSQLクラスタに関する講演が行われるという,
実に徹底した構成となっていました。

\section{理解を深めるチュートリアルトラック}

午後のチュートリアルトラックでは,PostgreSQL初級〜中級者がPostgreSQLをより深く理解するために,
PostgreSQL内部構成,システム設計の勘所,
そしてクエリプランの読み方のそれぞれについて深く掘り下げた講演が行われました。

その中でも特に個人的に面白かったのは,
NTT OSSセンタの坂本 昌彦氏による「実践!PostgreSQL 運用」です。
この講演では,運用そのものではなく,
運用時に問題が発生しないようなPostgreSQLシステムをいかに設計するかについて,
設計時のチェックポイントを紹介する形で行われました。
坂本氏曰く「とりあえずPostgreSQLを使えるのは当然として,
``プロ''としてプライドがある人ならばこのくらいできて欲しいよね,
という話をします」
という言葉通り,NTTにおける豊富なPostgreSQL運用経験をもとにして,
非常に具体的なところまで踏み込んだシステム設計ノウハウが説明されており,
こんな詳細なところまで喋っちゃっていいの!?
というくらい,PostgreSQLを使う上では役に立つ内容でした。

坂本氏の講演の中では,業務システムでPostgreSQLを利用する際の設定パラメータについて,
(Part 1) 接続数,(Part 2) メモリ設定,(Part 3) ディスク設定の3つにわけて,
チェックリスト形式で紹介されてゆきました。
いずれのパートでも,きっちりと性能を出すための設定は基本として,
想定される最悪のケースでも対処できるためにはどうすればよいか,
そのためのPostgreSQLの設定について徹底して検討された内容となっていました。
特に接続数の設定のところで,
Tomcat等のWebアプリケーション側からの接続数をどう見積もるかという話は,
ちょっと遊びでWebアプリケーションをかじっただけでは分からないような問題分析の知見が紹介されており,
まさにプロとしてのプライドを感じさせる内容の深さを感じました。

\section{気になる今後は}

今回のPgDayでは,PostgreSQL界隈で今まさにホットなレプリケーションが話題の中心で,
今後のPostgreSQLについてはほとんど語られていませんでした。
その部分について,現状で明らかになっているものに少しばかり触れてみましょう。

PgDayの基調講演で,Hagander氏は「新たな機能を提案するときに,
『ほかのDBMSにもあるから○○機能がPostgreSQLでも欲しい』
というのはやめてくれ」と言っていました。
しかしながら,今秋にリリースされたPostgreSQL 9.2の新機能や,
9.3でリリースが見込まれている新機能の情報を眺めていると,
やはり他のDBMSと比べて欠けているものを補おうとしている側面は見受けられます。
例えば,EnterpriseDB社\footnote{EnterpriseDB社にはコアチームメンバを始めとしたPostgreSQL開発の中心的人物が数多く所属している。}ではPostgreSQLの機能強化版であるPostgreSQL Plusを販売していますが,
PostgreSQL Plusの大きな特徴はOracleとの互換性,つまりOracleからの移行の容易性です。
Oracleと比べて足りない機能は,PostgreSQL開発ロードマップの重要な指針であることは間違いありません。

PostgreSQL 9.2の新機能であるIndex Only Scanは,
OracleやMySQLでは既に利用可能な機能であり,以前より強く待ち望まれていました。
\footnote{ちなみに,IBMのDB2では索引を張っていないカラムの集計もIndex Only Scanで出来るよう,
索引に指定した属性の値を同居させる機能を持っていたりします}
またPostgreSQL 9.0から次々に強化されるレプリケーション機能やPostgres-XCなどは,
明らかにOracle RACを意識したものでしょう。
9.2でのレプリケーションはあくまでもマスタは1インスタンスに限られていますが,
9.3ではマルチマスタ機能の導入も検討されているようです。
さらにPostgreSQL 9.3では,Materialized Viewが実装される見通しです。
これも商用DBでは既に一般的に提供されており,
PostgreSQLユーザからの要望調査第1位にある機能です\footnote{\url{http://postgresql.uservoice.com/forums/21853-general}}。

一方でPostgreSQL独特の面白い機能として,
書込み可能なfdw (Foreign Data Wrapper)のようなものもあります。
fdwを使うと,CSVファイルなどの外部データを,あたかもPostgreSQLのテーブルかのように扱い,
SQLを発行することができます。
現状ではデータの読込みしかできないのですが,
これを書込みもできるようにする機能の実装が進んでおり,
バージョン9.3での導入に向けて開発MLでも活発に議論されています。

PostgreSQLの開発そのものには関係ありませんが,
昨年から今年にかけてクラウドにおけるPostgreSQLの存在感が増してきています。
HerokuではDBaaSとして Heroku Postgres の提供を2011年末から開始しています。
最近では,データベースをgithubのレポジトリのようにforkできるという
面白い機能もリリースされて話題を呼びました\footnote{\url{https://postgres.heroku.com/blog/past/2012/11/8/fork_your_data/}}。
さらに,2010年末にHerokuを買収したSalesforceが,
PostgreSQLのメーリングリストにPostgreSQL技術者を募集するメールを投稿したことから,
SalesforceがOracleからOSSへシフトするのか?
と話題を呼びました\footnote{\url{http://archives.postgresql.org/pgsql-jobs/2012-10/msg00003.php}}。

また,つい最近になって富士通がPostgreSQLを搭載したデータベースアプライアンスを発表しました。
ここ数年のデータベース業界における1つの流れとして,
ハードとソフトを垂直統合したデータベースアプライアンスがよく話題になっているように感じます。
もともと古くからあった製品形態ではあるものの,
21世紀に入ってからは垂直統合から水平分業へと流れが傾きつつありました。
しかし,2009年にOracleが発表したExadataにより,
垂直統合のアプライアンスに再度注目が集まってきています。
その動きに呼応するかのように,2012年にはIBMもPure Systemsというアプライアンスを発表しています。
その市場にPostgreSQLまで加わるのか,という今回のニュースはちょっとした驚きでした。
業務利用が徐々に浸透しているPostgreSQLがどこまで闘えるのか注目です。

以上,雑多な感じでPostgreSQLの今後を占う情報を並べてみました。
着実に業務利用に耐えうるDBMSとして進化を続けていく一方で,
書き込み可能なfdwのようにOSSならではの面白い機能も追加されたりと,
PostgreSQLというソフトウェアの懐の深さ,今後の可能性が感じられたのではないかと思います。
