% -*- coding: utf-8 -*-

\chapter{PgDay 2012 Japan レポート}

\begin{flushright}
 {\headfont 著:はやみず}
\end{flushright}

2012年11月30日にNPO法人日本PostgreSQLユーザ会(JPUG)主催で開催されたPgDay 2012 Japanに参加してきました。

\section{開催概要}

日本におけるPgDayの名を冠するカンファレンスの開催は,今回が初めてだったようです。
JPUGの主催するPostgreSQLの大規模なカンファレンスは,
これまでは毎年2〜3月頃にPostgreSQL Conferenceが開催されていたのですが,
年度末で都合がつかずに参加できないという声が多いために,
秋開催へと移行することになっています。
そのため,今年は秋開催移行への予行演習的な意味も含めて?
2012年2回目のカンファレンスであるPgDay 2012 Japanが開催されました。

セミナーチケット140名分は早々と売り切れ,
追加分も販売され,当日の参加者数は222名(スタッフ発表)ということでした。
10月末に開催されたPG Conference Europe 2012では,
参加者がヨーロッパ全体から集まる4日間のカンファレンスで計290名参加であったことを考えると,
日本におけるPostgreSQL人気の高さ,
そしてユーザグループの成熟度がいかに高いかを伺うことができます。

\begin{figure}[tb]
 \begin{center}
  \includegraphics[width=12cm]{images/pgday-water.eps}\par
  \noindent 
{\bf ▲受付を済ませると,
  発表資料に加えてJPUGイベントではお馴染みの「かめすい」がもらえる}
 \end{center}
\end{figure}

\section{``You are PostgreSQL!''}

PgDay当日は,PostgreSQL開発チームのコアチームであるMagnus Hagander氏の基調講演で幕を開けました。
Hagander氏は ``Inside the PostgreSQL community - How the product gets built'' という題で,
PostgreSQLコミュニティにおける様々な人々の立ち位置や,
彼らがどのように協力しあってPostgreSQLを作り上げているのか,
そしてどのようにしてPostgreSQLの開発に参加できるか,
ということを紹介してゆきました。

この講演で一番強いメッセージとして感じたのは,
PostgreSQLは特定の企業には属しておらず,
そしてその開発は特定の企業や個人によってコントロールされるものではなく,
各々がコミュニティに参加して協力することで開発が進められている,
ということでした。
もちろんPostgreSQLでもEnterpriseDBや2ndQuadrant,
また日本からはNTTデータやSRA OSSなどの企業がPostgreSQLの開発に関わってはいるものの,
あくまでコミュニティに参加することを通して協力しながら開発を進めていることは,
他のDBMSとは一線を画します。
このような,様々な企業や個人が協力しながらコミュニティとして開発を進めていくことができる仕組み,
そして現在のPostgreSQLコミュニティを形作る人たちこそが,
継続的なOSSの進化・発展に不可欠な存在なのです。
つまり,ある意味PostgreSQLというプロダクトそのものよりも,
PostgreSQLというコミュニティのほうが重要であるというのが,
この講演における最大のメッセージと言えます。

そしてHagander氏のもうひとつの大きなメッセージは,
「今日この場にいる皆さん自身もPostgreSQLコミュニティの一員である(``You are PostgreSQL.'')」
ということでした。
PgDayに集まってきた参加者は,皆PostgreSQLに興味をもち,実際に使っている人たちであり,
PostgreSQLコミュニティの一員なのだ,ということです。
そして,更にPostgreSQLコミュニティの一員として積極的な参加を促すために,
Hagander氏は様々な形でPostgreSQLの開発に関わる方法を紹介してゆきました。
最も直接的にはパッチを書くという方法で参加することもできますが,
それ以外にもドキュメントを書いたり,翻訳したりという作業も「プロダクトを作る」
ということにおいては非常に重要であり,そして多くの人手を必要としています。
また,ユーザグループを作ったり,
PostgreSQLのことをブログに書くだけでも,
それが即ちPostgreSQLのマーケティング活動になるため,
価値あるPostgreSQLコミュニティ参加方法となります。
特に,最近では英語圏以外でもPostgreSQLに関する様々な情報発信が行われるようになっているため,
記事や書籍を英語に``逆輸入''することで,
世界中の人と知見を共有することができます。
このように,コードが書ける人も,書けない人も,
実に様々な側面からPostgreSQLコミュニティに参加することができる,
だから簡単なことから少しずつコミュニティに参加してみよう,
というHagander氏の熱い想いがこもった講演でした。

\section{注目は高可用データベースシステム}

午後からのプログラムは,PostgreSQLクラスタ特集といっても過言ではないほど,
レプリケーションや高可用技術に関する発表が中心でした。
なかでも,PostgreSQLのレプリケーション機能を中心とした高可用システムの解説や,
そのソリューションパッケージの紹介などが目立ちました。

PostgreSQLはこの数年,レプリケーション機能の充実に非常に力を入れています。
PostgreSQLでは,バージョン9.0から初めてレプリケーション機能が利用できるようになりました。
2010年にリリースされたPostgreSQL 9.0はもともと8.5となるはずでした。
しかし,レプリケーションとホットスタンバイという重大な2つの機能が新たに加わるインパクトを表すために,
あえてメジャーバージョンを増やして9.0になったという経緯があります。
このことからも,いかにレプリケーションが待ち望まれていたかがわかります。
その後も2011年にリリースされた PostgreSQL 9.1 では同期レプリケーション機能が,
2012年にリリースされたPostgreSQL 9.2ではカスケードレプリケーション機能が導入され,
用途・目的に応じて様々なレプリケーション方法を利用することができるようになりました。

PostgreSQLでレプリケーション機能が求められる大きな理由が,
業務用途,とくにサービスの停止時間ができるだけ短い,
高可用性データベースシステムにおいてPostgreSQLを使えるようにしたい,
という要望です。
これに応えるべくレプリケーション機能が進化したPostgreSQLの業務利用は確実に広まりをみせており,
日本国内でもその傾向は明らかに見て取れます。
JPUGの主催するPostgreSQLカンファレンスでは参加者数が年々増加しており,
また今秋にはPostgreSQLの業務システム適用を促進するためのPostgreSQLエンタープライズコンソーシアム(PGECons)が設立されました。

そのような背景もあり,
今年3月に開催されたPostgreSQL Conference 2012 Japanでも
レプリケーションや高可用システムの発表が目立ったのですが,
今回は更にその流れが色濃く反映されたプログラムとなっていました。
午後のプログラムはPostgreSQLのレプリケーション機能を紹介する講演,
そしてレプリケーション機能をベースとした高可用クラスタの構成に関するノウハウの講演が行われた後,
メイントラックとチュートリアルトラックに別れて,
メイントラックでは3本続けてPostgreSQLクラスタに関する講演が行われました。

\section{PostgreSQLへの理解を深めるチュートリアルトラック}

午後の2トラック構成では,
メイントラックでは注目のPostgreSQLクラスタに関する話題を取り扱っていた一方で,
チュートリアルトラックではPostgreSQL初級〜中級者がよりPostgreSQLを理解するために,
PostgreSQL内部構成,システム設計の勘所,
そしてクエリプランの読み方のそれぞれについて深く掘り下げた講演が行われました。

その中でも特に