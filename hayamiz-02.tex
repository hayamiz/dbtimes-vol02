% -*- coding: utf-8 -*-

\chapter{PgDay 2012 Japan レポート}

\begin{flushright}
 {\headfont 著:はやみず}
\end{flushright}

2012年11月30日にNPO法人日本PostgreSQLユーザ会(JPUG)主催で開催されたPgDay 2012 Japanに参加してきました。

\section{開催概要}

日本におけるPgDayの名を冠するカンファレンスの開催は,今回が初めてだったようです。
JPUGの主催するPostgreSQLの大規模なカンファレンスは,
これまでは毎年2〜3月頃にPostgreSQL Conferenceが開催されていたのですが,
年度末で都合がつかずに参加できないという声が多いために,
秋開催へと移行することになっています。
そのため,今年は秋開催移行への予行演習的な意味も含めて?
2012年2回目のカンファレンスであるPgDay 2012 Japanが開催されました。

セミナーチケット140名分は早々と売り切れ,
追加分も販売され,当日の参加者数は222名(スタッフ発表)ということでした。
10月末に開催されたPG Conference Europe 2012では,
参加者がヨーロッパ全体から集まる4日間のカンファレンスで計290名参加であったことを考えると,
日本におけるPostgreSQL人気の高さ,
そしてユーザグループの成熟度がいかに高いかを伺うことができます。

\section{プログラム構成}

