% -*- coding: utf-8 -*-

\chapter{Muninは舞い降りた ~データベースと俺とリソース監視~}
% * (アスタリスク)付きの \chapter* コマンドは原則不可とする

\begin{flushright}
 前佛 雅人(@zembutsu) % ペンネーム
\end{flushright}

\section{俺のデータベースがこんなに軽いはずがない}

これは、私がioDrive(NAND型フラッシュメモリ)搭載サーバで
ベンチマークを取得した際の印象である。
この業界では「重い」や「軽い」といった表現が
データベースやサーバの負荷状況を表す表現として用いられている。
しかし、この重さというのは、属人的な経験によるものが多いのではないだろうか。

本来、データベースやサーバなりシステムの性能を評価する際には
客観的なデータが必要である。データベースの性能なり
チューニングが適切だったかどうか、
あるいは、ハードウェア性能が適切だったかどうかも考慮しなくてはいけない。
その際に役立つツールが、Muninである。
Muninは様々なリソースの監視に特化したツールであり、
感覚的な重さを視覚化してるのに役立つツールなのである。

\section{俺とリソース監視}

\subsection{なぜリソース監視を行う必要があるのか}

% お風呂の水の例、監視対象の増加

\subsection{ホスティングの運用現場の話}


\section{Muninへの誘い}

\subsection{歴史}

\subsection{特長}

\subsection{他のツールとの違い}

\section{Muninプラグイン入門}

\subsection{プラグインの基本構造}

\subsection{プラグインの設定}

\section{RDBMS関連プラグインの活用}

\subsection{MySQL}

\subsection{PostgreSQL}

\section{カスタマイズへの道}

\subsection{プラグインを自作する}

